\documentclass[10pt,twoneside]{article}
%\documentclass[10pt,oneside]{book}
%\documentclass[14pt]{article}   	% use "amsart" instead of "article" for AMSLaTeX format
%\pagestyle{headings}

\usepackage{natbib}
\usepackage{hyperref}



\usepackage{titlesec}


\usepackage{enumitem}

%\usepackage[spanish]{babel}
 
\titleformat{\chapter}[display]
  {\normalfont\bfseries}{}{0pt}{\Huge}

\usepackage{blindtext}
\usepackage[T1]{fontenc}


\usepackage{bbm}

\usepackage[margin=1in,footskip=.25in]{geometry}


\usepackage{geometry}
\usepackage{mathtools}    
\usepackage[latin9]{inputenc}  	
\usepackage[bottom]{footmisc}	
\geometry{letterpaper}                   		
\usepackage{graphicx}														
\usepackage{amssymb}
\usepackage{ragged2e}
\newcommand{\kfour}{\mathbb{K}^4}
\newcommand{\ktwo}{\mathbb{K}^2}
\newcommand{\pthree}{\mathbb{P}^3}
\newcommand{\afin}{\mathbb{A}}
\newcommand{\ov}{\overrightarrow}


\usepackage{tikz-cd}


\usepackage{faktor}


\usepackage{mathtools}


%\usepackage[spanish]{babel} % español
%\usepackage[utf8]{inputenc} % acentos sin codigo
%\usepackage{graphicx} % graficos
%\usepackage{amssymb} 
%\usepackage{amsmath,amsthm} 
%\usepackage{amsthm} 

%\usepackage{hyperref}
%\usepackage{mathtools}
\usepackage{mathrsfs} % Letras caligráficas
\usepackage{bm}



\usepackage{xcolor}

\newcommand{\mcm}{\qopname \relax o{mcm}}

\newcommand{\massey}[3]{\langle[ {#1} ],[ {#2} ],[ {#3} ]\rangle}
\newcommand{\module}[1]{\vert #1 \lvert }
\newcommand{\dotprod}[2]{\langle #1,#2 \rangle}
\newcommand{\mbot}{{(\bot)}}
%\newcommand{\isoequal}{\stackrel{\text{iso}}{=}}
\newcommand{\isoequal}{\simeq}
\newcommand{\difequal}{\stackrel{\text{dif}}{=}}
\newcommand{\defequal}{\stackrel{\text{def}}{=}}
\newcommand{\isoapprox}{\stackrel{\text{iso}}{\displaystyle\approx}}
\newcommand{\Cech}{\v{C}ech}
\newcommand{\difpartial}[2]{\frac{\partial #1}{\partial #2}}
\newcommand{\difantipartial}[2]{\frac{\partial #1}{\overline{\partial} #2}}
\newcommand{\antipartial}{{\overline{\partial }}}

\newcommand{\comilla}{{}"{}}


\newcommand{\e}[2]{e_{#1}^{(#2)}}

\newcommand{\f}[2]{f_{#1}^{(#2)}}


\usepackage{amsthm}



\newcommand{\ifff}{if and only if }


\renewcommand\qedsymbol{$QED$}


%\footskip = 0pt

%\setlength{\parindent}{0cm}

\usepackage{titlesec}
\titleformat{\section}[block]{\color{black}\Large\bfseries\filcenter}{}{1em}{}

\date{}
\begin{document}

\section*{Exercise 1}

\subsection*{1}
\noindent

Let $x \in M$, then we can find a neighbourhood $U$ of $x$ and a chart $x:U\rightarrow \mathbb{R}^n$,
and $\{\frac{\partial}{\partial x_i}\}$ be the induced basis on $TU$.

Then locally, we can write:
$$
X=\sum x_i {\frac{\partial}{\partial x_i}}
;\quad
Y=\sum y_j {\frac{\partial}{\partial x_j}}
$$

Then a simple calculation yields:
$$
XYf
=
\sum x_i {\frac{\partial y_j}{\partial x_i}}{\frac{\partial f}{\partial x_j}}
+
\sum
x_i
y_j
\frac
{\partial^2 f}
{\partial x_j \partial x_i}
;\quad
YXf
=
\sum y_j {\frac{\partial x_i}{\partial x_j}}{\frac{\partial f}{\partial x_i}}
+
\sum
x_i
y_j
\frac
{\partial^2 f}
{\partial x_j \partial x_i}
$$

$$
[X,Y]
=
\sum \left(
x_i {\frac{\partial y_j}{\partial x_i}}
-
y_j {\frac{\partial x_i}{\partial x_i}}
\right)
\frac{\partial f}
{\partial x_j}
$$

Clearly, $B_x(X,Y)$ is symmetric \ifff $[X,Y]_x=0$. Since 
by definition of $B$,
$$B \cap U=
\left\{
x\in U
\left\vert
\frac
{\partial f}
{\partial x_j}
(x)
=
0
\right.
\right\}
$$
%$
%\frac
%{\partial f}
%{\partial x_j}
%(x)
%=
%0
%,
%\forall x\in B
%, \forall j \in\{1,\ldots,\text{dim } M\}$.
We get that if $x\in B$ then:

\begin{equation}
\label{eq:hessian1}
(XYf)_x
=
\sum
x_i
(x)
y_j
(x)
\frac
{\partial^2 f}
{\partial x_j \partial x_i}
(x)
;\quad
(YXf)_x
=
\sum
x_i(x)
y_j(x)
\frac
{\partial^2 f}
{\partial x_j \partial x_i}
(x)
;\quad
[X,Y]_x
=
0
\end{equation}

 Hence the symmetry of $B_x(X,Y)$ is proven and from \eqref{eq:hessian1} is clear that is only depends on $x_i(x)$ and
 $y_i(x)$, that is: $B_x(X,Y)$ only depends on $X_x$ and $Y_x$.
 
 Finally, if $B$ is a submanifold, then $X\in TB$, $X$
 can be integrated to obtain a curve $x(t)$ contained in B so that $X_{x(t)}=x'(t)$.
 
Since $x(t)\in B$ we have $\frac{\partial f}{\partial x_i}(x(t))=0$
, 
we can differentiate to obtain:
$$
% \frac{d}{dt}
0
=
%\sum x_i'(t)
%\frac{\partial f}{\partial x_i}
%+
\sum
_j
x_j'(t)
\frac{\partial^2 f}{\partial x_i x_j}
\quad,\forall i\in\{1,\ldots,n\}
$$

Since $x_i'(0)=x_i(x)$, by the above calculations we have that:

$$
(XYf)_x
=
\sum_{i,j=1}^n
x_i(x)
y_j(x)
\frac
{\partial^2 f}
{\partial x_j \partial x_i}
(x)
=
\sum_{j=1}^n
\left\{
y_j(x)
\sum_{i=1}^n
x_i(x)
\frac
{\partial^2 f}
{\partial x_j \partial x_i}
\right\}
=
0
$$

This proves the case $X_x\in T_xB$.
The case $Y_x\in T_xB$ is completely analogous, since
for $x\in B, (XYf)_x=(YXf)_x$
 
 \subsection*{2}
 \noindent
 
 Let $H_x(f)(u,v)\defequal (XYf)_x$, where we choose $X,Y$ so that $u=X_x$ and $v=Y_x$.

We have to check that the choice of $X,Y$ does not alter the value of H, however this was proven in the previous section where we checked that $(XYf)_x$ only depended on $X_x$ and $Y_x$.

Finally, since:
\begin{equation}
Ker H_x(f)
=
\{
X\in TM
\vert
H_x(f)(X,Y)=0,
\forall Y
\}
=
\{
X\in TM
\vert
(YXf)_x
=0
\forall Y
\}
=
TB
\end{equation}

we deduce that the hessian of f is non-degenerate on $N_{{B/M},x}$
. 

Finally, we can choose $U$ and a local frame 
$\mathcal{B}=\frac{\partial}{\partial x_j}$
on $U$ so that $H_x(f)$
has an associated symmetric matrix $M$
so that 

$$
H_x(f)
\left(\sum x_i\frac{\partial}{\partial x_i},
\sum y_j\frac{\partial}{\partial x_j}
\right)
=\sum_{i,j}x_iy_jM_{ij}
$$

Then if we change the frame
$\frac{\partial}{\partial x_j}$ to another frame
$\frac{\partial}{\partial y_j}$ we have a matrix $P$ 
a linear transformation on $TM$ so that
$P\frac{\partial}{\partial y_j}=\frac{\partial}{\partial x_j}$.

Then the matrix $M^{'}$ associated to the Hessian by the new matrix is given by:

$$
M^{'}=
P^TMP
$$

By Sylverster's inertia theorem,
we can find an invertible matrix $S$,
so that
$
S^TMS
$
is diagonal.

By the above discussion is enough to choose 
$\{S^{-1}\frac{\partial}{\partial x_i}\}$, 
as in this frame the Hessian has the form of $(0.5)$.

It also clear, that since each connected component $C_j$ of $B$
is a submanifold where $df\vert_{C_j}=0$ then $f\vert_{C_j}=const$.

\subsection*{3}

\begin{itemize}[label={}]

\item a)

By the usual Poincar� lemma, we can find a form $\gamma$ so that
$\gamma\vert_W=0$ and $d\gamma=\alpha$.

%By a theorem of the course, the inclusion $i:\Omega^k(M)^G\rightarrow\Omega^k(M)$
%is a quasi-isomomorphism. That is, we can find a equivariant form $\beta$,
%so that
%$\gamma-\beta$ is exact, in particular $d\beta=\alpha$. Furthermore, since the inclusion is linear,
%we must have that
%$\beta\vert_W=0$.
%
%So $\beta$ is the form we were looking for.

Since the action is trivial on $W$ we can find a 
$G-invariant$ neighbourhood of $W$
that we shall call $U$.
Then we apply the standard trick of averaging $\gamma$ over $U$. Let $\beta$
we defined as

$$
\beta
=
\int_U g^*\gamma d\mu
$$

Where we chose $\mu$ to be a Haar measure, as it is well-know that
compact Lie groups admits such a measure.
And we could normalize in the usual way by requiring
$
\beta
=
\int_U d\mu
=
1
$.

Then since the action is trivial on $W$, $\beta\vert_W=0$ and
since $d$ commutes with the integral sign and with the pullback 
by the naturality condition we have that:

$$
\begin{aligned}
d\beta
&=
d\int_U g^*\gamma d\mu
=
\int_U g^*(d\gamma) d\mu\\
&=
\int_U g^*\alpha d\mu
=
\int_U \alpha d\mu
\stackrel{(*)}{=}
\alpha \int_U d\mu\\
&=\alpha
\end{aligned}
$$

Where $(*)$ follows because $\alpha$
is invariant by hypothesis.

Furthermore, $\beta$ is invariant by construction.

\item b)

We follow the proof of the standard Darboux's lemma, with a slight twist:

Let $\omega_t=(1-t)\omega_0
+t\omega_1$. Then clearly, $\omega_t\vert_W=\omega_0\vert_W$.
Then let $X_t$ be an arbitrary differential vector field,
and let $\phi_t$ be it's flow. Then:

\begin{equation}
\begin{aligned}
\label{eq:moser}
\frac{d}{dt}\phi^*_t\omega_t
&=
\phi^*
(
L_{X_t}\omega_t
+
\omega_1
-
\omega_0
)
=0
\\
&\iff
L_{X_t}\omega_t
+
\omega_1
-
\omega_0
=0
\end{aligned}
\end{equation}

Then, because the action is trivial on $W$
we can find a neighbourhood $U$ of $W$
so that the action of $G$ is well defined. 
That is $g\cdot U
\subset U,\forall g\in G$.

By part a) we can, shrinking $U$ if necessary 
find a $G$-invariant form $\beta$
so that $d\beta=\omega_1-\omega_0$ and
$\beta_W=0$, since $\omega_1-\omega_0$
is in the hypothesis of a).

Then, using Cartan's formula, since $\omega_t$ is closed we get:
$L_{X_t}\omega_t=di_{X_t}\omega$.

So \eqref{eq:moser} turns into:
\begin{equation}
\label{eq:resb}
i_{X_t}\omega_t
+
\beta=0
\end{equation}

Because $\omega_t$ is not degenerate, because
it is the convex combination of two symplectic forms,
we can solve \eqref{eq:resb} to get a 
vector field $X_t$ that solves $\eqref{eq:moser}$,
since $\frac{d}{dt}\phi^*_t\omega_t=0
\implies
\phi^*_1\omega_1=\omega_0
$


If we could show that $\phi_t$
is $G$-invariant we would be done. We are going to show that
if we define $\hat{\phi}_t=g\cdot\phi_t(g^{-1}\cdot x)$. 
By the uniqueness guaranteed by the Picard-Lindel�f theorem we have to show that:

$$
\begin{cases}
\frac{\partial}{\partial t}\hat{\phi}_t=X_t\hat{\phi}_t\\
\hat{\phi}_0(x)=x
\end{cases}
$$

For the first part:

\begin{equation}
\begin{aligned}
\frac{\partial}{\partial t}\hat{\phi}_t
=
dg
X_t
\phi_t(g^{-1} \cdot x)
&=
dg
X_t
g^{-1}\cdot g
\phi_t(g^{-1} \cdot x)\\
&=
dg
X_t
g^{-1}
\hat{\phi_t}( x)
=
X_t
\hat{\phi_t}(x)
\end{aligned}
\end{equation}

Where last equality comes from the $G$-invariance of $X_t$
which in turn comes from the $G$-invariance of $\omega_t$.


For the second part: 
$
\hat{\phi}_0
=
g\cdot\phi_0(g^{-1}\cdot x)
=
g\cdot g^{-1} \cdot x
=x
$


And so the invariant Darboux theorem 
follows.

\end{itemize}


%\subsection*{4}

\section*{Exercise 2}

\subsection*{1}

We have that $\omega$ is invariant by hypothesis,
furthermore, since both M and $\mathbb{T}$ are compact we can
find $\langle\cdot,\cdot\rangle$ a bi-invariant scalar product
by the standard averaging procedure.

Then there is a unique matrix $A$ so that:

\begin{equation}
\label{eq:defA}
\omega(
X,
Y)
=
\langle
AX,Y
\rangle
,\quad
\forall X,Y\in TM
\end{equation}

Since both $\omega$ and $<\cdot,\cdot>$ are invariant, we have that by uniqueness,
so is $A$, that is because $g^{-1}Ag$ also verifies \eqref{eq:defA} as we show.
Let $g\in G$, then:

$$
\langle
AX,Y
\rangle
\stackrel{\eqref{eq:defA}}{=}
\omega(X,Y)
\stackrel{
*
}{=}
\omega(dg(X),dg(JY))
=
\langle
Adg(X),dg(Y)
\rangle
\stackrel{*}{=}
\langle
dg^{-1}Adg(X),Y
\rangle
$$

Where $*$ follows by the invariance of $\omega$ and $\langle \cdot,\cdot\rangle$

By the skew-symmetry of $\omega$ we have that $A=-A^T$, therefore $-A^2=AA^T$
is symmetric and we can take it's square root. (By diagonalising $-A^2=
Pdiag(\lambda_1,\ldots,\lambda_n)P^{-1}$, 
then $\sqrt{-A^2}=
Pdiag(\sqrt\lambda_1,\ldots,\sqrt\lambda_n)P^{-1}$)

Then 
$
J=
\left(
\sqrt{- A^2}
\right)^{-1}A
$
is invariant, because $A$ is
and is the composition of $G$-invariant homomorphism.

Clearly, $J^2=- A^{-2}
A^2=-Id$
and $J$ is compatible with $\omega$ because:

$$
\begin{aligned}
\omega(X,JX)
&=
\langle
AX,JX
\rangle\\
&=
\left\langle
AX,
\left(
\sqrt{- A^2}
\right)^{-1}A
X
\right\rangle
=
\left\langle
X,
\left(
\sqrt{- A^2}
\right)^{-1}
X
\right\rangle
>0
\end{aligned}
$$

Where the last inequality follows because $\sqrt{- A^2}^{-1}$
is definite positive.

$J$ is therefore the almost complex structure we were looking for.

\subsection*{2}

Since $g^{TM}$ is $G$-invariant,
$G$ acts by isometry, and takes geodesics to geodesics.
Let $\gamma(t,x_0,v_0)$
be the unique geodesic starting in $x_0$ with derivative in 0 $v_0$.
Then $exp_{x_0}(v_0)=\gamma(1,x_0,v_0)$

Therefore, if $x\in M^\mathbb{T}$, let's consider
 $\alpha(t)=g\cdot \gamma(t,x_0,v_0)$. By the above discussion, 
 $\alpha(t)$ is also a geodesic. 
 
 Since $\alpha(0))=g\cdot x_0=x_0$ and 
 $\alpha'(0)=dg_{x_0}(\gamma'(0,x_0,v_0))
 =dg_{x_0}(v_0)
 $
 
By uniqueness: $\alpha(t)=\gamma(t,x_0,dg_{x_0}(v_0))$.
In particular, for t=1 we get:

\begin{equation}
\label{eq:iso}
exp_{x_0}(dg_{x_0}(v_0))
=
g\cdot 
exp_{x_0}(v_0)
\end{equation}

\subsection*{3}

Let $x_0\in M^\mathbb{T}$,
then we can restrict the $U$
in the previous part to a $\mathbb{T}$-invariant neighbourhood.
In this way the $\mathbb{T}$-action on $U$ is well-defined.
Furthermore, since 
in the previous part we used an arbitrary 
$g\in \mathbb{T}$ we have that the exponential map is equivariant.

We notice that if 
$exp_{x_0}(v)\in M^\mathbb{T}$
then:
\begin{equation}
\label{eq:identify}
exp_{x_0}(v)
=
g\cdot exp_{x_0}(v)
=
exp_{x_0}(dg_{x_0}v),
\quad
\forall g\in \mathbb{T}
\end{equation}

This mens we can identify fixed points of the action in $U$
 with fixed point of the differential of the action on $T_{x_0}M$, since
 for $v\in T_{x_0}M, g\in\mathbb{T}$, $v=dg_{x_0}v$ because
 of \eqref{eq:identify} and $exp$ being an isomorphism.
 
That is, the exponential map send the eigenspace of $T_{x_0}M$ of eigenvalue 1 
to $M^\mathbb{T}\cap U$. Because the exponential map is a diffeomorphism it sends
submanifolds to submanifolds, in particular, it sends linear subspaces to submanifolds.

In summary:
Every point in $M^\mathbb{T}$
admits a neighbourhood that is parametrised by the exponential
map.

%Then we can restrict $U$ to a coordinate neighbourhood by a map
%$\phi$ so we have the following diagram:

%$$
%\begin{tikzcd}
%x^{-1}
%(
%\cap_{g\in\mathbb{T}} 
%\ker (Id-g)
%) \arrow{r}{d\phi_{\phi^{-1}(x_0)}}
%& T_{x_0}M \arrow{r}{exp_{x_0}}
%& C \arrow[r]
%& \cdots
%\end{tikzcd}
%$$

And so $M^\mathbb{T}$ is a submanifold.

\subsection*{4}


Because $x_0$ is a fixed point,
the action of $\mathbb{T}$ on $M$ lift to $T_{x_0}M$
via the differential, in particular, given $g,h\in\mathbb{T}$:

$$
d(gh)_{x_0}
=
dg_{x_0}dh_{x_0}
$$

By the chain rule. (We reiterate $x_0$ is a fixed point). 
Furthermore, since the action acts by symplectomorphism (by defintion),
we have that because $x_0$ is a fixed point if we choose
$g\in \mathbb{T}$:
$$
\omega_{x_0}(X,Y)
=
\omega_{g\cdot x_0}
(dg_{x_0}X,dg_{x_0}Y)
=
\omega_{x_0}
(dg_{x_0}X,dg_{x_0}Y)
$$

This proves that the action on $T_{x_0}M$
is symplectic. It only remains to prove that the action commutes with $J$,
however this is the definition for invariance of $J$ which we already prove.

\subsection*{5}

\subsection*{6}
We define $A_g\defequal \ker(\mathbbm{1}-dg_{x_0}g)$
to ease notation.

Let $v\in A_g$
we have that because of the previous parts
of the exercise:

\begin{equation}
\label{eq:sub}
v=
dg_{x_0}(J_{x_0}v)
=
J_{x_0}dg_{x_0}(v)
=
J_{x_0}v
\end{equation}

And so 
$J_{x_0}A_g=
A_g
$, because by \eqref{eq:sub} we have proved the $\subset$
inclusion and the other inclusion comes from the fact that
$J$ is an isomorphism since $-J$ is it's inverse.

From which we conclude that:

\begin{equation*}
T_{x_0}(M^\mathbb{T})
=
\left(
T_{x_0}M
\right)^\mathbb{T}
=
\bigcap_{g_\in\mathbb{T}}
A_g
\end{equation*}

If we showed that each of the $A_g$ was symplectic, then clearly
theirs intersection would be as well and the result would follow.

\bf{Claim}: $A_g$ is symplectic:

\begin{proof}
Let $g$ be an arbitrary element in $\mathbb{T}$
and $v\in A_g$ so that $i_v\omega_{x_0}=0$.
\end{proof}

\subsection*{7}

We have to prove two things:

\begin{enumerate}

\item $d(\mu,\tau)=i_\tau^{\mathbb{C}^n}\omega_{st}$.

\begin{proof}
We begin by seeing who is $\tau^{\mathbb{C}^n}$. 
Let $z=x+iy\in\mathbb{C}$ then:

\begin{align*}
\tau^{\mathbb{C}^n}\vert_z
=
\frac{d}{ds}
exp(s\tau)\cdot z
&=
(
-i\langle w_1,\tau\rangle z_1,
\ldots,
-i\langle w_n,\tau\rangle z_n
)
\\
&=
(
\langle w_1,\tau\rangle y_1,
\langle w_1,\tau\rangle x_1,
\ldots,
\langle w_1,\tau\rangle y_n,
\langle w_1,\tau\rangle x_n,
)\\
&=
\sum_{j=1}^n
\langle
w_j,\tau
\rangle
\left(
y_j\frac{\partial}{\partial x_j}
-x_j\frac{\partial}{\partial y_j}
\right)
\end{align*}

Again, by computation we get:

$$
i_{y_j\frac{\partial}{\partial x_j}
-x_j\frac{\partial}{\partial y_j}}
(dx_j\wedge dy_j)
=
x_jdx_j
+
y_jdy_j
$$

And gluing all together:
\begin{equation}
\label{eq:inclusion}
i_\tau^{\mathbb{C}^n}\omega_{st}
=
\sum_{j=1}^n
\langle
w_j,\tau
\rangle
(
x_jdx_j
+
y_jdy_j
)
\end{equation}

Now be go the LHS, let's see who is 
$d(\mu,\tau)$. Let $z\in\mathbb{C}$:

\begin{align}
\label{eq:momentum}
d(\mu_0(z),\tau)
=
d\left(
\frac{1}{2}
\sum_{j=1}^n
(x_j^2+y_j^2)
(w_j,\tau)
\right)
=
\sum_{j=1}^n
(x_jdx_j+y_jdy_j)
(w_j,\tau)
%=
%i_\tau^{\mathbb{C}^n}\omega_{st}
\end{align}

Joining \eqref{eq:inclusion} with \eqref{eq:momentum} we get the result.

\end{proof}

\item $\omega_{st}$ is equivariant, i.e
$\mu_0(g\cdot z)=Ad_{g}^*\mu_0(z)$.

\begin{proof}
Firstly, we notice that since $\mathbb{T}$ is abelian, so is it action, in the
sense that $Ad_{g}$ acts by identity. So the RHS is just $\mu(z)$.

For the LHS, we notice that since $\mathbb{T}$ is compact, it is generated as a group
by $\{exp(\tau),\tau\in\mathfrak{g}\}$, so if we show that 
$\mu_0(\exp(\tau)\cdot z)=\mu_0(z)$ this would that the LHS is also equal to $\mu_0(z)$.

We show the final claim:

$$
\mu_0(\exp(\tau)\cdot z)
=
\frac{1}{2}
\sum_{j=1}^n
\vert
e^{-i\langle
w_j,\tau
\rangle
}
z_j
\vert^2
w_j
=
\frac{1}{2}
\sum_{j=1}^n
\vert
z_j
\vert^2
w_j
=
\mu_0(z)
$$


\end{proof}

\end{enumerate}

\subsection*{8}

By definition of moment map we have:
$$
d(\mu,\tau)
=
i_{\tau^M}\omega
$$

And unraveling the identifications in $\mu_0$
we have that we actually have $\varphi^*\mu_0$,
where $\varphi$ is the diffeomorphism above.

$$
d(\varphi^*\mu_0,\tau)
=
\varphi^*i_{\tau^{\mathbb{C}^n}}\omega_{st}
=
i_{\varphi^*\tau^{\mathbb{C}^n}}\varphi^*\omega_{st}
=
i_{\varphi^*\tau^{\mathbb{C}^n}}\omega
$$

By the non-degeneracy of $\omega$ we have that
$$d(\mu-\mu_0)=0
\iff i_{\tau^M}\omega-i_{\varphi^*\tau^{\mathbb{C}^n}}\omega
\iff
\tau^M
=
\varphi^*\tau^{\mathbb{C}^n}
$$

We can show that the last equality holds by direct computation:
\begin{align*}
\varphi^*\tau^{\mathbb{C}^n}\vert_x
&=
d\varphi^{-1}\left(
\left.\frac{d}{ds}
\varphi
exp(s\tau)
\cdot 
x
\right\vert_{s=0}
\right)
\\
&=
d\varphi^{-1}\left(
d\varphi\left(
\left.\frac{d}{ds}
exp(s\tau)
\cdot 
x
\right\vert_{s=0}
\right)
\right)
\\
&=
\left.\frac{d}{ds}
exp(s\tau)
\cdot 
x
\right\vert_{s=0}
=
\left.\tau^M\right\vert_x
\end{align*}

Thus $d(\mu-\mu_0)=0$, 
which implies that $(\mu,\tau)-(\mu_0,\tau)$
is locally constant for any $\tau$. Since are working on a connected neighbourhood, 
we deduce that 
$
(\mu,\tau)=(\mu(x_0),\tau)+(\mu_0,\tau)
$.

Finally, since $\tau$ is arbitrary we conclude the result.

\subsection*{9}

By definition
$Crit(\mu_X)=
\{
x\in M:
d\mu_x
=0
\}
$
Then by the definition of momentum map we get: 
$$
\begin{aligned}
d(\mu_X)_x&=0\\
&\iff
d(\mu_X)_x(Y)=0,\quad \forall Y 
\in C^\infty(M,TM)\\
&\iff
\omega(X^M_x,Y)=0,
\quad \forall Y
\in C^\infty(M,TM)\\
&\iff
X^M_x=0
\end{aligned}
$$

The last equality follows from the 
non-degeneracy of $\omega$.

We notice that $X_x^M=0$ if and only if $x$ is a fixed point of the action, that is:

$$
Crit(\mu_x)
=
M^\mathbb{T}
$$

And so 
$Crit(\mu_x)$
is a symplectic submanifold because we have already proven it in $(6)$,

\subsection*{10}

It follows from the
previous section that,
since $M$ is compact, $Crit(\mu_X)$ has finitely many connected components, 
so they are open, and therefore submanifolds themselves.

The Hessian can be calculated as follows:
$
H_x(\mu_X)(Y,Z)
=
(ZY\mu_X)_x
$

Then:
$$
\begin{array}{l}
Y\mu_X =
d\mu_X(Y)=\omega(X^M,Y)\\
ZY\mu_X =
d(\omega(X^M,Y)(Z))
=
(i_{[Y,X^M]}\omega)(Z)
=
\omega
(
[Y,X^M]
,
Z
)
\end{array}
$$

It follows that:
$
H_x(\mu_X)(Y,Z)
=
\omega_x
(
[Y,X^M]
,
Z)
$.


\subsection*{11}


\section*{Exercise 3}

\subsection*{1}

By non-degeneracy of $\omega$
it is enough to show that for a given $X\in TM$
we have:

\begin{equation}
\label{eq:31}
\omega
\left(\mu^M,X
\right)
=
\omega
\left(-\frac{1}{2}
J(d\mathcal{H})^*))
,
X
\right)
\end{equation}

On the LHS we have:


On the RHS we have:

\begin{equation}
\begin{aligned}
\omega
\left(-\frac{1}{2}
J(d\mathcal{H})^*))
,
X
\right)
&=
-\frac{1}{2}
g^{TM}
\left(
X,
(d\mathcal{H})^*
\right)
\\
&=
-\frac{1}{2}
(d\mathcal{H})(X)
\\
&=
-\frac{1}{2}
X\langle
\mu(x),
\mu(x)
\rangle_\mathfrak{g}
\\
%&=
%-
%\langle
%X\mu(x),
%\mu(x)
%\rangle_\mathfrak{g}
\end{aligned}
\end{equation}

%Where the last equality follows from Leibniz rule, which 
%we can apply because $\mathfrak{g}$ is a f

From \eqref{eq:31} we get that $\mu^M_x=0\iff
d\mathcal{H}_x=0$
and so we get:
$$
Crit(\mathcal{H})
=
\{
x\in M:
\mu_x^M=0
\}
$$

As we wanted.

\subsection*{2}

Let $\mathcal{B}=\{
t_1,\ldots,t_n,
x_1,\ldots,x_m
\}$
be an orthonormal basis of
$\mathfrak{g}$ and 
$\mathcal{B}^*=\{
t_1^*,\ldots,t_n^*,
x_1^*,\ldots,x_m^*
\}$ it's dual basis of
$\mathfrak{g}^*$.
Notice that because of the orthonormality condition
the dual basis coincides in the usual sense and in the scalar product sense.

Then given $x\in M$, we have:
$$
\mu(x)
=
\sum_{i=1}^n
u_i
t_i
+
\sum_{j=1}^m
v_i
x_i
$$

Then because $x_i\vert_\mathfrak{t}=0$ we have:
$$
\mu_T(x)
=
\sum_{i=1}^n
u_i
t_i
=
P^\mathfrak{t}
\left(
\sum_{i=1}^n
u_i
t_i
+
\sum_{j=1}^m
v_i
x_i
\right)
=
P^\mathfrak{t}
\mu(x)
$$

\subsection*{3}

\subsection*{4}
This is a well-known theorem

\section*{Exercise 4}

\subsection*{1}

\subsection*{2}

\noindent

Obviously $\ker D_T\subset \ker D_T^2$. Let $x\in\ker D_T^2$,
then, because $D_T^2$ is self-adjoint:
$$
0=\langle
D_T^2x,x
\rangle
=
\langle
D_Tx,
D_Tx
\rangle
$$

So we conclude $D_Tx=0$ and so $\ker D_T^2\subset \ker D_T$
and $\ker D_T= \ker D_T^2$
\subsection*{3}

\subsection*{4}

\end{document}